\documentclass{article}
\usepackage[T1]{fontenc}
\usepackage[sfdefault, condensed]{roboto}
\usepackage[a4paper, width=135mmm, top=10mm, bottom=20mm]{geometry}
\usepackage{graphicx}
\usepackage{amsfonts,amsmath}
\numberwithin{equation}{section}
\newcommand{\source}[1]{\footnotesize\caption*{\textbf{Source}: {#1}} } %
\usepackage[dvipsnames]{xcolor}
\usepackage{booktabs}
\usepackage{graphicx}
\usepackage{threeparttable}
\usepackage{authblk}
\usepackage{subcaption}
\usepackage{textpos}
\usepackage{titling}
\usepackage{ragged2e}
\usepackage{relsize}
\usepackage{setspace}

\newlength{\templength}
\newlength{\textparindent}
\setlength{\textparindent}{\parindent}
\makeatletter
\let \@makefntextorig \@makefntext
% Saving the original definition so we can reuse it if necessary.
\newcommand{\@makefntextcustom}[1]{%
	\parindent \textparindent%
	\hspace{-\textparindent}\hspace{-100pt}\makebox[100pt][r]{\thefootnote.\enskip}%
	#1%
}

\renewcommand{\@makefntext}[1]{\@makefntextcustom{#1}}
\makeatother

\usepackage[colorlinks]{hyperref}
\hypersetup{
	linkcolor=blue,
	citecolor=blue,
	urlcolor=blue
}

\clubpenalty=100000
\widowpenalty=100000
\singlespacing
\usepackage{sectsty}
\allsectionsfont{\normalfont\sffamily\huge}

\begin{document}
	$if(title)$
	\maketitle
	$endif$
	$if(abstract)$
	\begin{abstract}
		$abstract$
	\end{abstract}
	$endif$
	
	$for(include-before)$
	$include-before$
	
	$endfor$
	$if(toc)$
	{
		$if(colorlinks)$
		\hypersetup{linkcolor=$if(toccolor)$$toccolor$$else$black$endif$}
		$endif$
		\setcounter{tocdepth}{$toc-depth$}
		\tableofcontents
	}
	$endif$
	$if(lot)$
	\listoftables
	$endif$
	$if(lof)$
	\listoffigures
	$endif$
	$body$
	
	$if(natbib)$
	$if(bibliography)$
	$if(biblio-title)$
	$if(book-class)$
	\renewcommand\bibname{$biblio-title$}
	$else$
	\renewcommand\refname{$biblio-title$}
	$endif$
	$endif$
	\bibliography{$for(bibliography)$$bibliography$$sep$,$endfor$}
	
	$endif$
	$endif$
	$if(biblatex)$
	\printbibliography$if(biblio-title)$[title=$biblio-title$]$endif$
	
	$endif$
	$for(include-after)$
	$include-after$
	
	$endfor$
\end{document}